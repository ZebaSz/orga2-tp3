\documentclass[10pt, a4paper, spanish]{article}
\usepackage[paper=a4paper, left=1.5cm, right=1.5cm, bottom=1.5cm, top=3.5cm]{geometry}
\usepackage[utf8]{inputenc}
\usepackage[spanish]{babel}
\usepackage{caratula}
\usepackage[pdfencoding=auto, colorlinks=true, linkcolor=blue]{hyperref}
\usepackage[boxruled, longend]{algorithm2e}
\usepackage{graphicx}
\usepackage{multirow}
\usepackage{wrapfig}
\usepackage{tikz}
\usepackage[rightcaption]{sidecap}
\usetikzlibrary{babel}

\tikzset{nodeList/.style={every node/.style={draw, circle}}}
\tikzset{pathList/.style={every node/.style={midway, fill=white}}}

\begin{document}

% CARATULA
\materia{Organización del Computador II}
\submateria{Primer Cuatrimestre de 2017}
\fecha{\today}
\grupo{Grupo ``InSisto gEnio zen acaba''}
\titulo{Trabajo Práctico 3}

\integrante{Bonggio, Enzo}{074/15}{ebonggio@dc.uba.ar}
\integrante{Szperling, Sebastián Ariel}{763/15}{sszperling@dc.uba.ar}
\integrante{Tarrío, Ignacio}{363/15}{itarrio@dc.uba.ar}

\maketitle

\tableofcontents

\pagebreak

\section{Ejercicio 1: GDT, Segmentos y Modo Protegido}

	\subsection{Descriptores de segmentos}

	Por las restricciones pedidas en el TP, nuestros descriptores comienzan en índice 8 de la GDT. El esquema utilizado es de segmentación \textit{flat}, con la excepción del segmento de video que solo es utilizado al dibujar el mapa por primera vez. PAra cumplir los requisitos, se definieron los segmentos como descriptores en la GDT de la siguiente manera:

	\begin{center}
		\begin{tabular}{ | c | c | c | c | c | c | }
		\hline
		Índice &                      Base &                   Límite &                  DPL & Tipo &                  G \\ \hline
		     8 & \multirow{4}{*}{0x000000} & \multirow{4}{*}{0x26EFF} & \multirow{2}{*}{0x0} &  0xA & \multirow{4}{*}{1} \\ \cline{5-5}
		     9 &                           &                          &                      &  0x2 &                    \\ \cline{4-5}
		    10 &                           &                          & \multirow{2}{*}{0x3} &  0xA &                    \\ \cline{5-5}
		    11 &                           &                          &                      &  0x2 &                    \\ \cline{2-6}
		    15 &                   0xB8000 &                   0x13F3 &                  0x0 &  0x2 &                  0 \\
		\hline
		\end{tabular}
	\end{center}

	Los niveles de privilegio corresponden a kernel (\texttt{0x0}) y a usuario/tarea (\texttt{0x3}), mientras que los tipos corresponden a código con lectura (\texttt{0xA}) y a datos con lectura/escritura (\texttt{0x2}).

	Todos los segmentos, además, están marcados como presentes (\texttt{p=1}) y de 32 bits (\texttt{db=1}). Los primeros 4 segmentos abarcan en total 623MB, mientras que el segmento de video solo ocupa lo necesario para escribir el buffer en su totalidad ($80 \times 50 \times 2$ bytes).

	\begin{figure}[h]
		\centering
		%\resizebox{\textwidth}{!}{
			\renewcommand{\arraystretch}{1.5}
			\begin{tabular}{ *{31}{|c}|c|}
				\multicolumn{1}{l}{31} & \multicolumn{6}{c}{} & \multicolumn{1}{r}{24} & \multicolumn{1}{c}{23} & \multicolumn{1}{c}{22} & \multicolumn{1}{c}{21} & \multicolumn{1}{c}{20} & \multicolumn{1}{l}{19} & \multicolumn{2}{c}{} & \multicolumn{1}{r}{16} & \multicolumn{1}{c}{15} & \multicolumn{1}{l}{14} & \multicolumn{1}{r}{13} & \multicolumn{1}{c}{12} & \multicolumn{1}{c}{11} & \multicolumn{2}{c}{} & \multicolumn{1}{c}{8} & \multicolumn{1}{l}{7} &\multicolumn{6}{c}{} & \multicolumn{1}{r}{0} \\
				\hline
				\multicolumn{8}{|c|}{0x00} & 1 & 1 & 0 & 0 & \multicolumn{4}{|c|}{0x2} & 1 & \multicolumn{2}{|c|}{0x2} & 1 & \multicolumn{4}{|c|}{0xA} & \multicolumn{8}{|c|}{0x00} \\
				\hline
				\multicolumn{32}{c}{} \\

				\multicolumn{1}{l}{31} & \multicolumn{14}{c}{} & \multicolumn{1}{r}{16} & \multicolumn{1}{l}{15} & \multicolumn{14}{c}{} & \multicolumn{1}{r}{0} \\
				\hline
				\multicolumn{16}{|c|}{0x0000} & \multicolumn{16}{|c|}{0x6EFF} \\
				\hline
			\end{tabular}
			\renewcommand{\arraystretch}{1}
		%}
		\caption{Ejemplo de descriptor de GDT: segmento de código de kernel}
	\end{figure}

	Dado que nuestros segmentos ocupan 623MB, antes de cargar la GDT debimos activar la linea A20.

	\subsection{Salto a modo protegido}

	Para pasar de modo real a modo protegido y configurar la pila del kernel, debimos:

	\lstset{escapechar=@,style=asm}
	\begin{enumerate}
		\item Setear el bit CR0.PE (bit 0):

		\begin{lstlisting}
mov eax, cr0
or eax, 1
mov cr0, eax
		\end{lstlisting}

		\item Inmediatamente realizar un jump far con selector 0x40 (índice 8 de la GDT, código de kernel):
    
		\begin{lstlisting}
jmp 0x40:modoprotegido
		\end{lstlisting}
		
		\item Configurar los registros de selectores con 0x48 (índice 9 de la GDT, datos de kernel), salvo por \texttt{fs} que fue seteado a 0x78 (índice 15 de la GDT, segmento de video):

		\begin{lstlisting}
mov ax, 1001000b ; index = 9 | gdt = 0 | rpl = 0
mov ds, ax
mov es, ax
mov gs, ax
mov ss, ax
mov ax, 1111000b ; index = 15 | gdt = 0 | rpl = 0
mov fs, ax
		\end{lstlisting}

		\item Por úlitmo, setear la pila a la dirección correspondiente:

		\begin{lstlisting}
mov ebp, 0x27000
mov esp, 0x27000
		\end{lstlisting}

	\end{enumerate}
\pagebreak
\section{Ejercicio 2: IDT y rutinas de atención a excepciones}
\pagebreak
\section{Ejercicio 3: Paginación y mapa de páginas del Kernel}

	\subsection{Mapeo de páginas del kernel}

	El kernel ocupa 1MB de memoria, con 3MB adicionales de area libre. Por ende, este area exactamente en el espacio de una tabla de páginas completa.

	Dado que todas las tareas deben tener este area mapeada igual que el kernel (con \textit{identity mapping}), creamos una función genérica para mappear el kernel que toma un page directory y un page table como parámetros:


	\lstset{escapechar=@,style=c}
	\begin{lstlisting}
void mmu_mapear_dir_kernel(unsigned int pd, unsigned int pt) {
	pd_entry* dir_paginas = (pd_entry*) pd;
	dir_paginas[0].p = 1; // bit presente
	dir_paginas[0].rw = 1; // incluye codigo y datos, read/write
	dir_paginas[0].us = 0; // area de kernel - privilegio supervisor
	dir_paginas[0].pwt = 0;
	dir_paginas[0].pcd = 0;
	dir_paginas[0].a = 0;
	dir_paginas[0].ign = 0;
	dir_paginas[0].ps = 0;
	dir_paginas[0].g = 0;
	dir_paginas[0].avl = 0;
	dir_paginas[0].page_addr = pt >> 12; // apunta a la base de la tabla de paginas

	pt_entry* tabla = (pt_entry*)pt;
	unsigned int i;
	for(i = 0; i < 1024; i++) {
		// mismos atributos que en la PDE
		tabla[i].p = 1;
		tabla[i].rw = 1;
		tabla[i].us = 0;
		tabla[i].pwt = 0;
		tabla[i].pcd = 0;
		tabla[i].a = 0;
		tabla[i].d = 0;
		tabla[i].pat = 0;
		tabla[i].g = 0;
		tabla[i].avl = 0;
		tabla[i].page_addr = i; // igual que el indice para identity mapping
	}
}
	\end{lstlisting}

	Esto asegura que todo el area del kernel está presente y mapeada por \textit{identity mapping} con permisos de supervisor. De este modo, para configurar los mapeos del kernel, alcanza con llamar a:

	\begin{lstlisting}
void mmu_inicializar_dir_kernel() {
	mmu_mapear_dir_kernel(0x27000, 0x28000);
}
	\end{lstlisting}

	siendo estas las direcciones del directorio y la tabla de páginas del kernel definidas por la cátedra.

	\subsection{Habilitado de paginación}

	Para habilitar el sistema de paginación, debimos:

	\lstset{escapechar=@,style=asm}
	\begin{enumerate}

		\item Inicializar el directorio de páginas del kernel como se explicó en el punto anterior:

		\begin{lstlisting}
call mmu_inicializar_dir_kernel
		\end{lstlisting}

		\item Cargar el directorio de páginas al registro CR3:

		\begin{lstlisting}
mov eax, 0x27000
mov cr3, eax
		\end{lstlisting}

		\item Setear el bit CR0.PG (bit 31):

		\begin{lstlisting}
mov eax, cr0
or eax, 0x80000000
mov cr0, eax
		\end{lstlisting}

	\end{enumerate}
\pagebreak
\section{Ejercicio 4: Unidad de Manejo de Memoria}

	\subsection{Inicialización}

	En la inicialización de la MMU, simplemente inicializamos el valor de la variable que nos dará la próxima página libre. Según el mapa de memoria de la cátedra, esta variable es inicializada al comienzo del area libre, ubicada en \texttt{0x100000}. Cada vez que se requiera una página libre, se retornará este valor y será avanzado en \texttt{0x1000} (para apuntar a la siguente página).

	Este area está mapeada con \textit{identity mapping}, por lo que no es necesario agregar las nuevas páginas al mappeo.

	\subsection{Mapeo de tareas ``zombi''}

	En la inicialización del mapeo de una tarea, debemos: 

	\begin{itemize}
		\item pedir dos páginas libres para utilizarlas como directorio y tabla de páginas y usarlas para todos los mapeos siguientes;

		\item mapear el area del kernel, correspondiende a los primeros 4MB de memoria;

		\item mapear el area del mapa en la que la tarea será copiada, que corresponde a la página de la tarea misma más sus casilleros aledaños;

		\item copiar el código de la tarea a su posición en el mapa, que requiere un mapeo temporal por \textit{identity mapping} en el directorio de páginas activo (es decir, el CR3 actual)
	\end{itemize}

	Para el mapeo del kernel podemos utilizar la función definida para el ejercicio 3, ya que el mapeo es idéntico (incluyendo los permisos de supervisor).

	Para realizar los mapeos del mapa, tanto los temporales como los que corresponden a la nueva tarea, se usan funciones que definimos a continuación para mapear páginas puntuales.

	En cuanto al area alrededor de la tarea/zombi:

	\begin{itemize}
		\item al calcular tanto la posición inicial como las posiciones relativas, se debe tener en cuenta el jugador que lanza la tarea:

		\lstset{escapechar=@,style=c}
		\begin{lstlisting}
char direccion = jugador == JUG_A ? 1 : -1;

unsigned int centro = yPos * MAP_MEM_WIDTH;
if(jugador == JUG_B) {
	centro += MAP_MEM_WIDTH - PAGE_SIZE;
}
		\end{lstlisting}

		\item en el mapeo inicial solo se mapean 5 de los 8 casilleros adyacentes (ya que el zombi se encuentra al borde del mapa, no tiene sus casilleros ``anteriores'');

		\item para las posiciones superiores e inferiores (o izquierda y derecha del zombi) se debe tener en cuenta el caso en que la dirección excede los límites del mapa, y debe mapearse el extremo opuesto en caso contrario; por ejemplo:

		\begin{lstlisting}
mmu_mapear_pagina(TASK_VIRT+(2*PAGE_SIZE), pd,
	MAP_START + mem_mod(centro + direccion * (PAGE_SIZE + MAP_MEM_WIDTH), MAP_MEM_SIZE));
		\end{lstlisting}

		donde la función \texttt{mem\_mod} simplemente calcula el módulo entre dos números (a diferencia del operador \verb|%| que puede dar resultados negativos).

	\end{itemize}

	\subsection{Mapeo general de páginas}

	Para el mappeo de páginas definimos 2 macros que nos permiten calcular los offsets en el directorio y la tabla correspondientes:

	\begin{lstlisting}
#define PDE_OFFSET(virtual) virtual >> 22
#define PTE_OFFSET(virtual) (virtual << 10) >> 22
	\end{lstlisting}

	Por otro lado, creamos 2 funciones para crear o destruir mappeos para usuario:
	\begin{itemize}

		\item Para la creación de un mapeo, primero se revisa la PDE correspondiente: si la tabla no está marcada como presente, se pide una nueva página para esta ser utilizada como tabla de páginas y se la almacena como presente en la PDE mencionada. Luego, en esta tabla se accede a la entrada correspondiente y se guarda la base de la dirección física con atributos de lecto-escritura y nivel de privilegio usuario y el bit de presente.

		\item Para desmappear, el procedimiento es muy similar, pero tras conseguir la dirección de la PTE, basta con limpiar el bit de presente en la misma; si la tabla de páginas no existe (no está presente), la función no hace nada.
	\end{itemize}

	Luego de ambas funciones se llama a la función \texttt{tlbflush()}, que se encarga de limpiar el caché de traducciones de direcciones (\textit{Translation Lookaside Buffer}) para que el cambio se vea reflejado en caso de tratarse del CR3 actual.

	Cabe aclarar que las tareas necesitan mapear 1 página con privilegios de supervisor (la pila de nivel 0 correspondiente a la misma), por lo cual esto se corrige luego del mapeo. Esto no genera conflictos con el caché de traducciones, ya que esta página se mapea una única vez durante la creación de la tarea (durante la cual su CR3 no es el activo).
\pagebreak
\section{Ejercicio 5: Interrupciones de reloj y teclado}

	\subsection{Rutina de atención a reloj}

	Dentro de la rutina del reloj se encuentra el código encargado de la conmutación de tareas:

    \lstset{escapechar=@,style=asm}
	\begin{lstlisting}
sched_tarea_offset:     dd 0x00
sched_tarea_selector:   dw 0x00

_isr32:
    pushad
    test byte [ENDGAME], 1
    jnz .end

    call proximo_reloj

    call sched_tarea_actual
    cmp eax, 16 
    jge .next_task
    push eax
    call game_print_clock
    pop eax

    .next_task

    call sched_proximo_indice
    cmp ax, 0
    je .nojump
        mov [sched_tarea_selector], ax
        call fin_intr_pic1
        jmp far [sched_tarea_offset]
        jmp .end

    .nojump:
    call fin_intr_pic1

    .end:
    popad
    iret
	\end{lstlisting}

	Para lograr saber a que tarea saltar debe llamarse al scheduler, explicado en mayor detalle en el ejercicio 7, el cual nos devuelve el selector de la próxima tarea a ejecutar. En caso de que no haya que realizar un cambio de tarea (no hay tareas o se trata de la tarea actual) volvemos con \textit{IRET}.

	En este punto también actualizamos el reloj de la tarea que se ejecuta actualmente en la pantalla, además del reloj global del sistema. Si el juego terminó, el sistema deja de conmutar tareas.

	Por último, nos ocupamos de resetear el PIC para poder recibir las siguientes interrupción. Esto lo realizamos siempre, más allá de si realiza un cambio de tarea o no.

	\subsection{Rutina de atención a teclado}

	Las interrupciones del teclado nos dan la posibilidad de poder elegir y lanzar tareas, pero también nos permiten activar y desactivar el modo debug del cual daremos mas detalles en el ejercicio 7.

	Tenemos en cuenta dos eventos a la hora de activar o desactivar la posibilidad de lanzar tareas o elegirlas, estas son: 
	\begin{itemize}
		\item el final de juego la cual anula por completo la posibilidad de lanzar o cambiar el zombi.

		\item y ademas cuando el cartel de debug se esta mostrando no se permite hacer ninguna acción, salvo desactivar el cartel para poder seguir jugando.
	\end{itemize}

	No se tomo en cuenta los códigos de tecla para cuando uno suelta una tecla, ya que solo consideramos como casos a resolver por la interrupción los códigos de las teclas \textit{A, S, W, X, L\_Shift} para el \texttt{jugador A} y \textit{J, K, I, M, R\_Shift} para el \texttt{jugador B}, ademas contamos con la tecla \textit{Y} para activar el modo debug caso contrario la interrupcion no realiza ninguna acción.

	Como con la interrupción del reloj, se debe resetear el PIC para poder recibir nuevamente una interrupción.
\pagebreak
\section{Ejercicio 6: TSSs y salto a tarea Idle}
\pagebreak
\section{Ejercicio 7: scheduling, servicio mover y modo debug}
    
    \subsection{Inicialización del scheduler}

    Para hacer funcionar el scheduler debimos asegurarnos de tener las interrupciones activadas con el PIC configurado para poder recibir las interrupciones del reloj. Esto se inicializa antes de saltar a la tarea Idle por primera vez.

    Adicionalmente, el scheduler se inicializa con tareas inválida, ya que el mismo mantiene el número de la última tarea ejecutada para cada jugador. Los valores inválidos omiten ciertos procesos como el dibujado del reloj de la tarea actual (la tarea Idle maneja su propio reloj).

	\subsection{Servicio mover}
	Las tareas cuentan con una única manera de comunicarse con el kernel, esto lo logran realizando una interrupción a \texttt{0x66} indicando en \textit{eax} hacia donde quiere moverse la tarea.

	Para lograr esto dentro de la rutina de atención a la interrupción realizamos las siguientes acciones:

	\begin{itemize}
		\item Pintar el rastro del zombi, es decir el lugar donde esta actualmente con un (*).

		\item Sabiendo que tarea se esta ejecutando actualmente podemos saber que zombi es y a quien pertenece. Ademas sabemos hacia donde se va a mover chequeando el registro \textit{eax}. Utilizar dicha información para dibujar la nueva ubicacion del zombie.

		\item Chequear si las condiciones para anotar un punto estan dadas. Si esto es asi se procede a matar a la tarea siguiendo los pasos ya indicados en el ejercicio 2.

		\item Caso contrario mapear y desmapear las paginas del zombi. Primero desmapear el area actual y luego mapear las nuevas paginas.

		\item Saltar a idle.
	\end{itemize}

	Luego de que termine el ciclo de clock en idle el scheduler se encargara de devolver la próxima tarea a ejecutar.

	\subsection{Modo debug}
	Si el modo debug se encuentra activado (mediante la tecla \textit{Y}) el juego pasará a mostrar la siguiente excepción que se produzca en pantalla. Para lograr dicho cometido en la sección de manejo de excepciones se guarda de la información de todos los registros de uso común, los segmentos, la excepción que se disparo. Solo cuando el modo debug esta activado entonces esta información es pasada al juego el cual la almacena para poder usarla en un paso posterior.

	Guardamos la información del mapa y mostramos la información de la excepción en un paso intermedio entre indicarle al scheduler que la tarea no se encuentra mas activa y saltar a idle (los pasos que se realizan al eliminar una tarea se encuentran en el ejercicio 2).

	Salir del modo debug:

	\begin{lstlisting}

_isr33:

    in al, 0x60
    cmp al, key_debug
    je .toggle_debug

    test byte [debug_flag], debug_shown
    jnz .keyboard_end

    ...

    .toggle_debug:
        mov al, [debug_flag]
        test al, (debug_shown | debug_on)
        jz .enable_debug
        test al, debug_shown
        jz .keyboard_end

        ; disable_debug
        mov byte [debug_flag], debug_off
        call sched_toggle_debug ; indicar al scheduler que debe conmutar tareas nuevamente
        call game_debug_close ; restaurar el estado del mapa pre-debug
        jmp .keyboard_end

        .enable_debug:
        mov byte [debug_flag], debug_on
        jmp .keyboard_end
    ...
	\end{lstlisting}

	Para poder salir del modo debug y continuar el juego debemos presionar nuevamente la tecla \textit{Y}, cuando lo hacemos dibujamos nuevamente la pantalla con la información que estaba guardada y el scheduler buscara la próxima tarea a ejecutar.
\pagebreak
\section{Apéndice: modificaciones a funciones provistas por la cátedra}


% compilar 2 veces para actualizar las referencias


\end{document}