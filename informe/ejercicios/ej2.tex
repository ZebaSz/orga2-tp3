\section{Ejercicio 2: IDT y rutinas de atención a excepciones}

	\subsection{Inicialización}

	Para poder tener acceso a la \textit{IDT} se requiere de un descriptor que indique el lugar donde se encuentra el arreglo con las entradas de atención a interrupciones. 

	\begin{lstlisting}
		idt_descriptor IDT_DESC = {
	    	sizeof(idt) - 1,
	    	(unsigned int) &idt
		};
	\end{lstlisting}

	Las entradas de la \textit{IDT} fueron generadas mediante la utilización del macro provisto por la cátedra, el cual especificaba que la atención de la misma iba a darse en el archivos \textit{isr.asm} en el método llamado \textit{\_isrX} donde \textit{X} representa el numero de interrupción a responder. 

	El único cambio significativo que realizamos en la generación de las entradas de la \textit{IDT} es modificar el atributo de la interrupción \texttt{0x66}, ya que esta iba a funcionar a modo de syscall para la tarea, por ende el código debía ser accesible desde nivel 3. 

	Una vez llenados los campos procedimos a activar la \textit{IDT} de la siguiente manera :

	\begin{lstlisting}
   		; Cargar IDT
    		lidt [IDT_DESC] ; IDT_DESC es el descriptor arriba mencionado
	\end{lstlisting}

	\subsection{Excepciones}

	Podemos dividir a las interrupciones y hablar del subgrupo excepciones que son errores generados por el \textit{CPU}. En muchos casos las excepciones no son realmente errores.

	Para atender las excepciones usamos el macro provisto por la cátedra:

	\begin{lstlisting}
   		%macro ISR 1
		global _isr%1

		_isr%1:
		    ;xchg bx, bx
		    push %1
		    jmp matar_tarea

		%endmacro

		ISR 0; inicializamos los macros que eran necesarios para atender las excepciones
		...
	\end{lstlisting}

	En cuando al código de manejo de la excepción agregamos la lógica de matar a la tarea y saltar a idle. A lo que nos referimos cuando hablamos de matar a la tarea:

	\begin{itemize}
		\item{Guardar datos relevantes sobre el estado de los registros para en caso de ser necesario mostrarlos en pantalla (modo debug).}
		\item{Imprimir en pantalla la muerte del zombi.}
		\item{Chequear si las condiciones para que el juego finalice se cumplen. Es decir ver si aun quedan zombis por tirar.}
		\item{En caso de estar en modo debug mostrar la información anteriormente guardada.}
		\item{Indicarle al scheduler que la tarea no esta activa.}
		\item{Saltar a idle.}
	\end{itemize} 

	De esta manera al próximo ciclo de clock el scheduler se encargara nuevamente de buscar la tarea correspondiente.