\section{Ejercicio 2: IDT y rutinas de atención a excepciones}

	\subsection{Inicialización}

	Para poder tener acceso a la \textit{IDT} se requiere de un descriptor que indique el lugar donde se encuentra el arreglo con las entradas de atención a interrupciones. 

	\begin{lstlisting}
idt_descriptor IDT_DESC = {
	sizeof(idt) - 1,
	(unsigned int) &idt
};
	\end{lstlisting}

	Las entradas de la \textit{IDT} fueron generadas mediante la utilización del macro provisto por la cátedra, el cual especificaba que la atención de la misma iba a darse en el archivos \textit{isr.asm} en el método llamado \textit{\_isrX} donde \textit{X} representa el numero de interrupción a responder. Se completaron todas las entradas correspondientes a las excepciones definidas por Intel, además de las interrupciones de reloj (32) y teclado (33) y la puerta de servicio de sistema (102 o \texttt{0x66}).

	Todas utilizan el selector de segmento de código del kernel (\texttt{0x40}) y cuentan con atributo \texttt{0x8E00} (interrupt gates presentes de nivel 0), salvo por la interrupción \texttt{0x66} que debe ser accesible desde nivel de privilegio usuario (3), por lo que su atributo se modifica a \texttt{0xEE00}.

	Una vez llenados los campos cargamos la IDT a través de la instrucción \texttt{lidt [IDT\_DESC]}.

	\subsection{Excepciones}

	En el caso de una excepción, nuestro sistema debe desalojar la tarea actual y saltar nuevamente a la tarea Idle. Para atender las excepciones usamos el macro provisto por la cátedra:

	\begin{lstlisting}
%macro ISR 1
global _isr%1

_isr%1:
    push %1
    jmp matar_tarea

%endmacro

ISR 0; inicializamos los macros que eran necesarios para atender las excepciones
...
	\end{lstlisting}

	En cuando al código de manejo de la excepción agregamos la lógica de matar a la tarea y saltar a idle. A lo que nos referimos cuando hablamos de matar a la tarea:

	\begin{itemize}
		\item Guardar datos relevantes sobre el estado de los registros para en caso de ser necesario mostrarlos en pantalla (modo debug).

		\item Imprimir en pantalla la muerte del zombi.

		\item Chequear si las condiciones para que el juego finalice se cumplen. Es decir ver si aun quedan zombis por tirar.

		\item En caso de estar en modo debug mostrar la información anteriormente guardada.

		\item Indicarle al scheduler que la tarea no esta activa.

		\item Saltar a idle.
	\end{itemize} 

	De esta manera al próximo ciclo de clock el scheduler se encargara nuevamente de buscar la tarea correspondiente.