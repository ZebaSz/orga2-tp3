\section{Ejercicio 3: Paginación y mapa de páginas del Kernel}

	\subsection{Mapeo de páginas del kernel}

	El kernel ocupa 1MB de memoria, con 3MB adicionales de area libre. Por ende, este area exactamente en el espacio de una tabla de páginas completa.

	Dado que todas las tareas deben tener este area mapeada igual que el kernel (con \textit{identity mapping}), creamos una función genérica para mappear el kernel que toma un page directory y un page table como parámetros:


	\lstset{escapechar=@,style=c}
	\begin{lstlisting}
void mmu_mapear_dir_kernel(unsigned int pd, unsigned int pt) {
	pd_entry* dir_paginas = (pd_entry*) pd;
	dir_paginas[0].p = 1; // bit presente
	dir_paginas[0].rw = 1; // incluye codigo y datos, read/write
	dir_paginas[0].us = 0; // area de kernel - privilegio supervisor
	dir_paginas[0].pwt = 0;
	dir_paginas[0].pcd = 0;
	dir_paginas[0].a = 0;
	dir_paginas[0].ign = 0;
	dir_paginas[0].ps = 0;
	dir_paginas[0].g = 0;
	dir_paginas[0].avl = 0;
	dir_paginas[0].page_addr = pt >> 12; // apunta a la base de la tabla de paginas

	pt_entry* tabla = (pt_entry*)pt;
	unsigned int i;
	for(i = 0; i < 1024; i++) {
		// mismos atributos que en la PDE
		tabla[i].p = 1;
		tabla[i].rw = 1;
		tabla[i].us = 0;
		tabla[i].pwt = 0;
		tabla[i].pcd = 0;
		tabla[i].a = 0;
		tabla[i].d = 0;
		tabla[i].pat = 0;
		tabla[i].g = 0;
		tabla[i].avl = 0;
		tabla[i].page_addr = i; // igual que el indice para identity mapping
	}
}
	\end{lstlisting}

	Esto asegura que todo el area del kernel está presente y mapeada por \textit{identity mapping} con permisos de supervisor. De este modo, para configurar los mapeos del kernel, alcanza con llamar a:

	\begin{lstlisting}
void mmu_inicializar_dir_kernel() {
	mmu_mapear_dir_kernel(0x27000, 0x28000);
}
	\end{lstlisting}

	siendo estas las direcciones del directorio y la tabla de páginas del kernel definidas por la cátedra.

	\subsection{Habilitado de paginación}

	Para habilitar el sistema de paginación, debimos:

	\lstset{escapechar=@,style=asm}
	\begin{enumerate}

		\item Inicializar el directorio de páginas del kernel como se explicó en el punto anterior:

		\begin{lstlisting}
call mmu_inicializar_dir_kernel
		\end{lstlisting}

		\item Cargar el directorio de páginas al registro CR3:

		\begin{lstlisting}
mov eax, 0x27000
mov cr3, eax
		\end{lstlisting}

		\item Setear el bit CR0.PG (bit 31):

		\begin{lstlisting}
mov eax, cr0
or eax, 0x80000000
mov cr0, eax
		\end{lstlisting}

	\end{enumerate}