\section{Ejercicio 5: Interrupciones de reloj y teclado}

	\subsection{Rutina de atención a reloj}
	Dentro de la rutina del reloj se encuentra el código que proporcionan la concurrencia de tareas.

	\begin{lstlisting}
	sched_tarea_offset:     dd 0x00
	sched_tarea_selector:   dw 0x00

	_isr32:
	    pushad
	    test byte [ENDGAME], 1
	    jnz .end

    	call proximo_reloj

	    call sched_tarea_actual
	    cmp eax, 16 
	    jge .next_task
	    push eax
	    call game_print_clock
	    pop eax

	    .next_task

	    call sched_proximo_indice 
	    ;xor eax, eax
	    cmp ax, 0
	    je .nojump
	        mov [sched_tarea_selector], ax
	        call fin_intr_pic1
	        jmp far [sched_tarea_offset]
	        jmp .end

	    .nojump:
	    call fin_intr_pic1

	    .end:
	    popad
	    iret
	\end{lstlisting}

	Para lograr la concurrencia realizamos una llamada al scheduler, explicado en mayor detalle en el ejercicio 6, el cual nos devuelve cual sera la próxima tarea a ejecutar. En caso de que no halla que realizar un cambio de tarea volvemos con \textit{IRET}. 

	Ademas también el reloj deja de realizar la concurrencia entre tareas si el juego terminó. En este punto también actualizamos el reloj de la tarea que se ejecuta actualmente en la pantalla. 

	Por ultimo, nos ocupamos de resetear el pic para poder recibir otra interrupción de lejos. Esto lo realizamos sea en caso de cambiar de tarea o de permanecer en la misma.

	\subsection{Rutina de atención a teclado}
	Las interrupciones del teclado nos dan la posibilidad de poder elegir y lanzar tareas, pero también nos permiten activar y desactivar el modo debug del cual daremos mas detalles en el ejercicio 7.

	Tenemos en cuenta dos eventos a la hora de activar o desactivar la posibilidad de lanzar tareas o elegirlas, estas son: 
	\begin{itemize}
		\item{el final de juego la cual anula por completo la posibilidad de lanzar o cambiar el zombi.}

		\item{y ademas cuando el cartel de debug se esta mostrando no se permite hacer ninguna acción, salvo desactivar el cartel para poder seguir jugando.}
	\end{itemize}

	No se tomo en cuenta los códigos de tecla para cuando uno suelta una tecla, ya que solo consideramos como casos a resolver por la interrupción los códigos de las teclas \textit{A, S, W, X, lShift} para el \texttt{jugador A} y \textit{J, K, I, M, rShift} para el \texttt{jugador B}, ademas contamos con la tecla \textit{Y} para activar el modo debug caso contrario la interrupcion no realiza ninguna acción.

	Al final de la interrupción se debe resetear el pic para poder recibir nuevamente una interrupción mediante el siguiente call a función :

	\begin{lstlisting}
		extern fin_intr_pic1
		...
   		.keyboard_end:
    		call fin_intr_pic1
    		...	
	\end{lstlisting}
