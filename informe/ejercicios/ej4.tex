\section{Ejercicio 4: Unidad de Manejo de Memoria}

	\subsection{Inicialización}

	En la inicialización de la MMU, simplemente inicializamos el valor de la variable que nos dará la próxima página libre. Según el mapa de memoria de la cátedra, esta variable es inicializada al comienzo del area libre, ubicada en \texttt{0x100000}. Cada vez que se requiera una página libre, se retornará este valor y será avanzado en \texttt{0x1000} (para apuntar a la siguente página).

	Este area está mapeada con \textit{identity mapping}, por lo que no es necesario agregar las nuevas páginas al mappeo.

	\subsection{Mapeo de tareas ``zombi''}

	En la inicialización del mapeo de una tarea, debemos: 

	\begin{itemize}
		\item pedir dos páginas libres para utilizarlas como directorio y tabla de páginas y usarlas para todos los mapeos siguientes;

		\item mapear el area del kernel, correspondiende a los primeros 4MB de memoria;

		\item mapear el area del mapa en la que la tarea será copiada, que corresponde a la página de la tarea misma más sus casilleros aledaños;

		\item copiar el código de la tarea a su posición en el mapa, que requiere un mapeo temporal por \textit{identity mapping} en el directorio de páginas activo (es decir, el CR3 actual)
	\end{itemize}

	Para el mapeo del kernel podemos utilizar la función definida para el ejercicio 3, ya que el mapeo es idéntico (incluyendo los permisos de supervisor).

	Para realizar los mapeos del mapa, tanto los temporales como los que corresponden a la nueva tarea, se usan funciones que definimos a continuación para mapear páginas puntuales.

	En cuanto al area alrededor de la tarea/zombi:

	\begin{itemize}
		\item al calcular tanto la posición inicial como las posiciones relativas, se debe tener en cuenta el jugador que lanza la tarea:

		\lstset{escapechar=@,style=c}
		\begin{lstlisting}
char direccion = jugador == JUG_A ? 1 : -1;

unsigned int centro = yPos * MAP_MEM_WIDTH;
if(jugador == JUG_B) {
	centro += MAP_MEM_WIDTH - PAGE_SIZE;
}
		\end{lstlisting}

		\item en el mapeo inicial solo se mapean 5 de los 8 casilleros adyacentes (ya que el zombi se encuentra al borde del mapa, no tiene sus casilleros ``anteriores'');

		\item para las posiciones superiores e inferiores (o izquierda y derecha del zombi) se debe tener en cuenta el caso en que la dirección excede los límites del mapa, y debe mapearse el extremo opuesto en caso contrario; por ejemplo:

		\begin{lstlisting}
mmu_mapear_pagina(TASK_VIRT+(2*PAGE_SIZE), pd,
	MAP_START + mem_mod(centro + direccion * (PAGE_SIZE + MAP_MEM_WIDTH), MAP_MEM_SIZE));
		\end{lstlisting}

		donde la función \texttt{mem\_mod} simplemente calcula el módulo entre dos números (a diferencia del operador \verb|%| que puede dar resultados negativos).

	\end{itemize}

	\subsection{Mapeo general de páginas}

	Para el mappeo de páginas definimos 2 macros que nos permiten calcular los offsets en el directorio y la tabla correspondientes:

	\begin{lstlisting}
#define PDE_OFFSET(virtual) virtual >> 22
#define PTE_OFFSET(virtual) (virtual << 10) >> 22
	\end{lstlisting}

	Por otro lado, creamos 2 funciones para crear o destruir mappeos para usuario:
	\begin{itemize}

		\item Para la creación de un mapeo, primero se revisa la PDE correspondiente: si la tabla no está marcada como presente, se pide una nueva página para esta ser utilizada como tabla de páginas y se la almacena como presente en la PDE mencionada. Luego, en esta tabla se accede a la entrada correspondiente y se guarda la base de la dirección física con atributos de lecto-escritura y nivel de privilegio usuario y el bit de presente.

		\item Para desmappear, el procedimiento es muy similar, pero tras conseguir la dirección de la PTE, basta con limpiar el bit de presente en la misma; si la tabla de páginas no existe (no está presente), la función no hace nada.
	\end{itemize}

	Luego de ambas funciones se llama a la función \texttt{tlbflush()}, que se encarga de limpiar el caché de traducciones de direcciones (\textit{Translation Lookaside Buffer}) para que el cambio se vea reflejado en caso de tratarse del CR3 actual.

	Cabe aclarar que las tareas necesitan mapear 1 página con privilegios de supervisor (la pila de nivel 0 correspondiente a la misma), por lo cual esto se corrige luego del mapeo. Esto no genera conflictos con el caché de traducciones, ya que esta página se mapea una única vez durante la creación de la tarea (durante la cual su CR3 no es el activo).