\section{Ejercicio 7: scheduling, servicio mover y modo debug}


\subsection{Servicio mover}
	Las tareas cuentan con una unica manera de comunicarse con el kernel, esto lo logran realizando una interrupcion a \texttt{0x66} indicando  en eax hacia donde quiere moverse la tarea.

	Para lograr esto dentro de la rutina de atencion a la interrupcion realizamos las siguientes acciones:

	\begin{itemize}
		\item{Pintar el rastro del zombie, es decir el lugar donde esta actualmente con un (*).}
		\item{Sabiendo que tarea se esta ejecutando actualmente podemos saber que zombie es y a quien pertenece. Utilizar dicha informacion para dibujar la nueva ubicacion del zombie.}
		\item{Chequear si las condiciones para anotar un punto estan dadas. Si esto es asi se procede a matar a la tarea siguiendo los pasos ya indicados en el ejercicio 2.}
		\item{Caso contrario mapear y desmapear las paginas del zombie. Primero desmapear el area actual y luego mapear las nuevas paginas.}
		\item{Saltar a idle.}
	\end{itemize}

	Luego de que termine el ciclo de clock en idle el scheduler se encargara de devolver la proxima tarea a ejecutar.

\subsection{Modo debug}	