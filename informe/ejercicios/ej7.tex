\section{Ejercicio 7: scheduling, servicio mover y modo debug}
    
    \subsection{Inicialización del scheduler}

    Para hacer funcionar el scheduler debimos asegurarnos de tener las interrupciones activadas con el PIC configurado para poder recibir las interrupciones del reloj. Esto se inicializa antes de saltar a la tarea Idle por primera vez.

    Adicionalmente, el scheduler se inicializa con tareas inválida, ya que el mismo mantiene el número de la última tarea ejecutada para cada jugador. Los valores inválidos omiten ciertos procesos como el dibujado del reloj de la tarea actual (la tarea Idle maneja su propio reloj).

    \subsection{Conmutación de tareas}

    El scheduler guarda varios datos sobre el estado de las tareas: por un lado, tiene un arreglo con el estado de las 16 tareas (en ejecución o detenidas); por el otro, mantiene en particular las últimas tareas de cada jugador, para poder asignarles tiempo en ronda.

    También guarda la última tarea ejecutada para mantener este registro correctamente y es usada por otros servicios del sistema. Este valor puede no coincidir con la tarea indicada en el \texttt{tr}, ya que la tarea es desalojada al moverse, pero debemos respetar el orden de ejecución de todos modos.

    Para identificar la siguiente tarea a ejecutar, el scheduler busca una tarea activa sobre el arreglo de tareas la siguiente tarea activa:
    
    \lstset{escapechar=@,style=c}
    \begin{lstlisting}
unsigned int sched_buscar_tarea(unsigned int jugador, unsigned int status) {
    unsigned int tarea = tareasAnteriores[jugador] + 1;
    if(tarea >= (TASK_PER_PLAYER * (1 + jugador))) {
        tarea = TASK_PER_PLAYER * jugador;
    }
    int i;
    for (i = 0; i < TASK_PER_PLAYER && status_tareas[tarea] != status; ++i) {
        ++tarea;
        if(tarea >= (TASK_PER_PLAYER * (1 + jugador))) {
            tarea = TASK_PER_PLAYER * jugador;
        }
    }
    return tarea;
}
    \end{lstlisting}

    Si la tarea retornada por este método es igual a la anterior, significa que el jugador posee como máximo 1 tarea activa.

    Si la tarea retornada está inactiva, el jugador no tiene tareas activas, por lo que se busca una tarea del jugador actual. Si el mismo tampoco tiene tareas, o tiene una única tarea (y la misma coincide con el \texttt{tr}, es decir, no fue desalojada), no hace falta conmutar

    Esta lógica también es utilizada para buscar una tarea libre en caso que deba lanzarse un zombi nuevo, por lo que no se asigna el primer zombi libre, sino que se asignan en secuencia. Esto podría modificarse tomando la tarea anterior por parámetro en lugar de buscarla en el arreglo dentro de este método.

    El scheduler también provee la siguiente funcionalidad:

    \begin{enumerate}

        \item Una función para lanzar una tarea, que retorna su número asignado y la marca como iniciada

        \begin{lstlisting}
unsigned int sched_lanzar_tarea(unsigned int jugador) {
    unsigned int nuevaTarea = sched_buscar_tarea(jugador, FALSE);
    status_tareas[nuevaTarea] = TRUE;
    return nuevaTarea;
}
        \end{lstlisting}

        \item Una función para terminar la tarea en ejecución, que marca la misma como finalizada y salta a la tarea Idle (utilizando un jump far)

        \begin{lstlisting}
void sched_matar_tarea_actual() {
    status_tareas[tareaActual] = FALSE;
    __asm __volatile("ljmp $0x68, $0" : : );
}
        \end{lstlisting}

        \item Un toggle para el modo debug, ya que no se debe conmutar tareas durante el mismo


    \end{enumerate}

	\subsection{Servicio mover}
	Las tareas cuentan con una única manera de comunicarse con el kernel, esto lo logran realizando una interrupción a \texttt{0x66} indicando en \textit{eax} hacia donde quiere moverse la tarea.

	Para lograr esto dentro de la rutina de atención a la interrupción realizamos las siguientes acciones:

    \lstset{escapechar=@,style=c}
	\begin{enumerate}
		\item Pintar el rastro del zombi, es decir el lugar donde esta actualmente con un \textit{(*)}.
\begin{lstlisting}
game_print_rastro(zombis[tarea].xPos, zombis[tarea].yPos);
\end{lstlisting}



		\item Chequear si las condiciones para anotar un punto estan dadas. Si esto es asi se procede a matar a la tarea siguiendo los pasos ya indicados en el ejercicio 2.
\begin{lstlisting}
if(newXPos == 0 || newXPos == 79) {
    unsigned int puntoPara = newXPos == 0 ? JUG_B : JUG_A;
    if(++jugadores[puntoPara].score == 10) {
        game_finalizar();
    }
    game_print_score(puntoPara);
    game_matar_zombi_actual();
    sched_matar_tarea_actual();
}
\end{lstlisting}

		\item Caso contrario mapear y desmapear las paginas del zombi. Primero desmapear el area actual y luego mapear las nuevas paginas.
\begin{lstlisting}
mmu_mover_zombi(owner, zombis[tarea].xPos-1, zombis[tarea].yPos-1, newXPos-1, newYPos-1);
\end{lstlisting}

        \item Sabiendo que tarea se esta ejecutando actualmente podemos saber que zombi es y a quien pertenece. Ademas sabemos hacia donde se va a mover chequeando el registro \textit{eax}. Utilizar dicha información para dibujar la nueva ubicacion del zombie.
\begin{lstlisting}
game_print_zombi_mapa(tarea);
\end{lstlisting}

		\item Saltar a idle.
\begin{lstlisting}
mov [sched_tarea_selector], word 0x68
jmp far [sched_tarea_offset]
\end{lstlisting}        
	\end{enumerate}

	Luego de que termine el ciclo de clock en idle el scheduler se encargara de devolver la próxima tarea a ejecutar.

	\subsection{Modo debug}
	Si el modo debug se encuentra activado (mediante la tecla \textit{Y}) el juego pasará a mostrar la siguiente excepción que se produzca en pantalla. Para lograr dicho cometido en la sección de manejo de excepciones se guarda de la información de todos los registros de uso común, los segmentos, la excepción que se disparo. Solo cuando el modo debug esta activado entonces esta información es pasada al juego el cual la almacena para poder usarla en un paso posterior.

	Guardamos la información del mapa y mostramos la información de la excepción en un paso intermedio entre indicarle al scheduler que la tarea no se encuentra mas activa y saltar a idle (los pasos que se realizan al eliminar una tarea se encuentran en el ejercicio 2).

	Salir del modo debug:

	\begin{lstlisting}

_isr33:

    in al, 0x60
    cmp al, key_debug
    je .toggle_debug

    test byte [debug_flag], debug_shown
    jnz .keyboard_end

    ...

    .toggle_debug:
        mov al, [debug_flag]
        test al, (debug_shown | debug_on)
        jz .enable_debug
        test al, debug_shown
        jz .keyboard_end

        ; disable_debug
        mov byte [debug_flag], debug_off
        call sched_toggle_debug ; indicar al scheduler que debe conmutar tareas nuevamente
        call game_debug_close ; restaurar el estado del mapa pre-debug
        jmp .keyboard_end

        .enable_debug:
        mov byte [debug_flag], debug_on
        jmp .keyboard_end
    ...
	\end{lstlisting}

	Para poder salir del modo debug y continuar el juego debemos presionar nuevamente la tecla \textit{Y}, cuando lo hacemos dibujamos nuevamente la pantalla con la información que estaba guardada y el scheduler buscara la próxima tarea a ejecutar.