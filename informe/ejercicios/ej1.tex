\section{Ejercicio 1: GDT, Segmentos y Modo Protegido}

	\subsection{Descriptores de segmentos}

	Por las restricciones pedidas en el TP, nuestros descriptores comienzan en índice 8 de la GDT. El esquema utilizado es de segmentación \textit{flat}, con la excepción del segmento de video que solo es utilizado al dibujar el mapa por primera vez. PAra cumplir los requisitos, se definieron los segmentos como descriptores en la GDT de la siguiente manera:

	\begin{center}
		\begin{tabular}{ | c | c | c | c | c | c | }
		\hline
		Índice &                      Base &                   Límite &                  DPL & Tipo &                  G \\ \hline
		     8 & \multirow{4}{*}{0x000000} & \multirow{4}{*}{0x26EFF} & \multirow{2}{*}{0x0} &  0xA & \multirow{4}{*}{1} \\ \cline{5-5}
		     9 &                           &                          &                      &  0x2 &                    \\ \cline{4-5}
		    10 &                           &                          & \multirow{2}{*}{0x3} &  0xA &                    \\ \cline{5-5}
		    11 &                           &                          &                      &  0x2 &                    \\ \cline{2-6}
		    15 &                   0xB8000 &                   0x13F3 &                  0x0 &  0x2 &                  0 \\
		\hline
		\end{tabular}
	\end{center}

	Los niveles de privilegio corresponden a kernel (\texttt{0x0}) y a usuario/tarea (\texttt{0x3}), mientras que los tipos corresponden a código con lectura (\texttt{0xA}) y a datos con lectura/escritura (\texttt{0x2}).

	Todos los segmentos, además, están marcados como presentes (\texttt{p=1}) y de 32 bits (\texttt{db=1}). Los primeros 4 segmentos abarcan en total 623MB, mientras que el segmento de video solo ocupa lo necesario para escribir el buffer en su totalidad ($80 \times 50 \times 2$ bytes).

	\begin{figure}[h]
		\centering
		%\resizebox{\textwidth}{!}{
			\renewcommand{\arraystretch}{1.5}
			\begin{tabular}{ *{31}{|c}|c|}
				\multicolumn{1}{l}{31} & \multicolumn{6}{c}{} & \multicolumn{1}{r}{24} & \multicolumn{1}{c}{23} & \multicolumn{1}{c}{22} & \multicolumn{1}{c}{21} & \multicolumn{1}{c}{20} & \multicolumn{1}{l}{19} & \multicolumn{2}{c}{} & \multicolumn{1}{r}{16} & \multicolumn{1}{c}{15} & \multicolumn{1}{l}{14} & \multicolumn{1}{r}{13} & \multicolumn{1}{c}{12} & \multicolumn{1}{c}{11} & \multicolumn{2}{c}{} & \multicolumn{1}{c}{8} & \multicolumn{1}{l}{7} &\multicolumn{6}{c}{} & \multicolumn{1}{r}{0} \\
				\hline
				\multicolumn{8}{|c|}{0x00} & 1 & 1 & 0 & 0 & \multicolumn{4}{|c|}{0x2} & 1 & \multicolumn{2}{|c|}{0x2} & 1 & \multicolumn{4}{|c|}{0xA} & \multicolumn{8}{|c|}{0x00} \\
				\hline
				\multicolumn{32}{c}{} \\

				\multicolumn{1}{l}{31} & \multicolumn{14}{c}{} & \multicolumn{1}{r}{16} & \multicolumn{1}{l}{15} & \multicolumn{14}{c}{} & \multicolumn{1}{r}{0} \\
				\hline
				\multicolumn{16}{|c|}{0x0000} & \multicolumn{16}{|c|}{0x6EFF} \\
				\hline
			\end{tabular}
			\renewcommand{\arraystretch}{1}
		%}
		\caption{Ejemplo de descriptor de GDT: segmento de código de kernel}
	\end{figure}

	Dado que nuestros segmentos ocupan 623MB, antes de cargar la GDT debimos activar la linea A20.

	\subsection{Salto a modo protegido}

	Para pasar de modo real a modo protegido y configurar la pila del kernel, debimos:

	\begin{enumerate}
		\item Setear el bit CR0.PE:

		\texttt{mov eax, cr0}\\
		\texttt{or eax, 1}\\
		\texttt{mov cr0, eax}

		\item Inmediatamente realizar un jump far con selector 0x40 (índice 8 de la GDT, código de kernel):
    
		\texttt{jmp 0x40:modoprotegido}
		
		\item Configurar los registros de selectores con 0x48 (índice 9 de la GDT, datos de kernel), salvo por \texttt{fs} que fue seteado a 0x78 (índice 15 de la GDT, segmento de video):

		\texttt{xor eax, eax}\\
		\texttt{mov ax, 1001000b ; index = 9 | gdt = 0 | rpl = 0}\\
		\texttt{mov ds, ax}\\
		\texttt{mov es, ax}\\
		\texttt{mov gs, ax}\\
		\texttt{mov ss, ax}\\
		\texttt{mov ax, 1111000b ; index = 15 | gdt = 0 | rpl = 0}
		\texttt{mov fs, ax}

		\item Por úlitmo, settear la pila a la dirección correspondiente:

		\texttt{mov ebp, 0x27000}\\
		\texttt{mov esp, 0x27000}

	\end{enumerate}