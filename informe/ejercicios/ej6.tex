\section{Ejercicio 6: TSSs y salto a tarea Idle}
Dado que el sistema corre concurrentemente varias tareas, necesitamos alguna forma de mantener el contexto de las tareas que estén en pausa. Para esto, podemos guardar en la GDT los descriptores de TSS para cada una zombie y una para la tarea idle y definir una TSS a la que apuntan estos descriptores, aunque estas no se van a inicializar hasta que se cree la tarea. También hay que tener en cuenta que cuando saltamos por primera vez a una tarea el procesador va a intentar guardar el contexto por lo que tenemos que crear una TSS para este salto, notemos que el contenido del contexto no nos interesa.

Para la tarea idle se reservó el índice 13 de la GDT, y la tss de la tarea se configuró con la pila apuntando a donde se encuentra el kernel, el cr3 del kernel y el eip apuntando a donde se encuentra la tarea, en cuanto al descriptor se colocó el DPL en 0 y al ocupar un tamaño menor a 1mb, el bit G también se colocó en 0.

Al inicializar las tareas zombies vamos a inicializar el descriptor de TSS de la tarea con DPL nivel 3 y apuntará a la tss que nos otorga el arreglo de TSSs, según qué número de tarea para el jugador. En cuanto a la TSS, el CR3 está dado por lo que retorna la función de inicialización de la memoria del zombie, para los selectores de segmento DS, ES, FS, GS, los seteamos como el selector de datos nivel usuario, el CS como selector de código nivel usuario, como queremos que las interrupciones estén habilitadas, vamos a setear los flags con el bit correspondiente activo (0x202).
