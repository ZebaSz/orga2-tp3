\section{Ejercicio 6: TSSs y salto a tarea Idle}

	\subsection{Entradas en la GDT}

	Para saltar a las distintas tareas, definimos un total de 18 entradas en la GDT correspondientes a cada TSS. Las mismas tienen todas los mismos atributos: son de sistema, tipo 0x9, con límite 0x68. La dirección base de cada TSS está definida por 2 posiciones constantes (para la inicial y la Idle) y 2 dos arreglos (para las tareas zombi).

	Para que todas las entradas entren cómodamente en la GDT, aumentamos su tamaño a 32. Sin embargo, podría reducirse a 30 si 1) usamos un espacio en blanco que dejamos para linear los números y 2) utilizamos la TSS de una tarea zombi como TSS inicial.

	\subsection{Inicialización de las TSSs}

	La TSS de la tarea inicial no hace falta inicializarla ya que solo utiliza como dummy para poder saltar a la "siguiente" (primera) tarea, la Idle. Por ende, sus valores nunca son leídos, solo escritos.

	La TSS Idle se inicializa con los siguientes valores:

	\begin{itemize}
		\item \texttt{CS = 0x40} (segmento de código del kernel)

		\item \texttt{DS, ES, FS, GS, SS, SS0 = 0x48} (segmento de datos del kernel)

		\item \texttt{ESP, EBP, ESP0 = 0x27000} (base de pila del kernel)

		\item \texttt{CR3 = 0x27000} (directorio de páginas del kernel)

		\item \texttt{EIP = 0x10000} (base de la tarea Idle)
	\end{itemize}

	En cuanto a las tareas zombi, sus TSSs se inicializan de la siguiente manera:

	\begin{itemize}
		\item \texttt{CS = 0x53} (segmento de código de usuario, con DPL 3)

		\item \texttt{DS, ES, FS, GS, SS = 0x5B} (segmento de datos de usuario, con DPL 3)

		\item \texttt{ESP, EBP = 0x08001000} (límite de página virtual de la tarea)

		\item \texttt{SS0 = 0x48} (segmento de datos del kernel)

		\item \texttt{ESP0 = 0x0800b000} (límite de dirección virtual mapeada a pagina nueva)

		\item \texttt{CR3 = mmu\_inicializar\_dir\_zombi} (directorio creado durante la inicialización, ver sección 4.2)

		\item \texttt{EIP = 0x08000000} (base de página virtual de la tarea)

	\end{itemize}

	La base de la página designada para el stack de nivel 0 (ESP0) se ubica en \texttt{0x0800a000}.

	Para todas las tareas, los registros generales se inicializan en 0, el mapa de I/0 en \texttt{0xFFFF} y los \texttt{EFLAGS} en \texttt{0x202} (interrupciones activadas).

	\subsection{Salto a tarea Idle}

	Para saltar a la primera tarea debimos:

	\lstset{escapechar=@,style=asm}
	\begin{enumerate}

		\item Inicializar las TSSs y agregarlas a la GDT:

		\begin{lstlisting}
call tss_inicializar
call tss_inicializar_idle
		\end{lstlisting}

		\item Cargar la TSS inicial como "tarea actual":

		\begin{lstlisting}
mov ax, 0x60
ltr ax
		\end{lstlisting}

		\item Hacer un jump far (task switch) a la tarea Idle:

		\begin{lstlisting}
jmp 0x68:0
		\end{lstlisting}

	\end{enumerate}